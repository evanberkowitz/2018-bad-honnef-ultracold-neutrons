%%%
%%% Match requested formatting
%%%

\documentclass[a4paper,12pt]{article}                               % A4 with 12 point font
\usepackage[left=25mm,right=25mm,top=30mm,bottom=25mm]{geometry}    % Margins ...
\tolerance=5000                                                     % ... which MUST be respected.
\renewcommand{\familydefault}{\sfdefault}                           % Helvetica


%%%
%%% My 
%%%

\usepackage{amsmath,amssymb}
\usepackage{bm}
\usepackage[dvipsnames]{xcolor}
\usepackage[
    colorlinks=true,
    allcolors=black
]{hyperref}

%%%
%%%     Draft margin.
%%%

\usepackage[
    angle=90,
    color=black,
    opacity=1,
    scale=2,
    ]{background}
\SetBgPosition{current page.west}
\SetBgVshift{-4.5mm}
\backgroundsetup{contents={\input{git_information}}}

\usepackage{xspace}
\usepackage{bbm}

%%%%
%%%%    Referring to Parts of the Document
%%%%

\newcommand{\tabref}[1]{Tab.~\ref{tab:#1}\xspace}
\newcommand{\Tabref}[1]{Table~\ref{tab:#1}\xspace}
\newcommand{\figref}[1]{Fig.~\ref{fig:#1}\xspace}
\newcommand{\Figref}[1]{Figure~\ref{fig:#1}\xspace}
\renewcommand{\eqref}[1]{(\ref{#1})\xspace}
\newcommand{\Eqref}[1]{Equation~\ref{eq:#1}\xspace}

%%%%
%%%%    Mathematical Symbols
%%%%

\newcommand{\goesto}{\ensuremath{\rightarrow}}
\newcommand{\one}{\ensuremath{\mathbbm{1}}}

%%%%
%%%%    Physical Quantities and Constants
%%%%


%%%%
%%%%    Software
%%%%

\newcommand{\git}{\texttt{git}\xspace}

% Put an xspace after \LaTeX:
\let\builtinLaTeX\LaTeX
\def\LaTeX{\builtinLaTeX\xspace}
 % input rather than include so we don't create macros.aux

%%%%
%%%%    Document preparation
%%%%


\begin{document}

\title{The Nucleon Axial Coupling from QCD}

\author{\underline{Evan Berkowitz}\\ The CalLat Collaboration}
\date{\small \it
    Institut f\"{u}r Kernphysik \& Institute for Advanced Simulation \\ Forschungszentrum J\"{u}lich \\ 52425 J\"{u}lich, Germany
}

\maketitle

The axial coupling of the nucleon, $g_A=1.2723(23)$, is the strength of its coupling to the weak axial current of the Standard Model.
It dictates the rate at which neutrons decay into protons, the strength of the attractive long-range force between nucleons, and other features of nuclear physics.
To disentangle Standard Model effects from new physics, precision tests of the Standard Model in nuclear environments require a quantitative understanding of nuclear physics rooted in QCD.

The nucleon axial coupling has long been considered a critical benchmark for lattice QCD, and yet proved substantially more challenging to calculate than expected.
I will describe our recent percent-level calculation, $g_A=1.271(13)$, and how our method differs from existing results.
I will discuss what is required for a calculation that can discriminate between existing experimental results.


\vspace{16pt}
{\large \bf References}
\vspace{16pt}

\setlength{\parindent}{0cm}
\setlength{\parskip}{16pt}
[1] C.C. Chang \emph{et al.}, ``A per-cent-level determination of the nucleon axial coupling from Quantum Chromodynamics'', \emph{Nature}, 558:91-94, 2018, DOI:\href{https://doi.org/10.1038/s41586-018-0161-8}{10.1038/s41586-018-0161-8}.  \href{https://arxiv.org/abs/1805.12130}{arXiv:1805.12130}.

[2] Berkowitz \emph{et al.}, ``An Accurate Calculation of the Nucleon Axial Charge with Lattice QCD'', 2017.  \href{https://arxiv.org/abs/1704.01114}{arXiv:1704.01114}.

% \cite{2017:ga_nature}
%
% \bibliographystyle{unsrt}
% \bibliography{master}

\end{document}