\documentclass[aps,superscriptaddress,tightenlines,nofootinbib,floatfix,longbibliography]{revtex4-1}
\usepackage[left=18mm,right=19mm,top=23mm,bottom=16mm]{geometry}
\usepackage{amsmath,amssymb}
\usepackage{bm}
\usepackage{comment}
\usepackage{graphicx}
\usepackage[dvipsnames]{xcolor}
\usepackage{slashed}
\usepackage{hyperref}


%%%
%%%     Draft margin.
%%%

\usepackage[
    angle=90,
    color=red,
    opacity=1,
    scale=2,
    ]{background}
\SetBgPosition{current page.west}
\SetBgVshift{-4.5mm}
\backgroundsetup{contents={\input{git_information}}}

\usepackage{xspace}
\usepackage{bbm}

%%%%
%%%%    Referring to Parts of the Document
%%%%

\newcommand{\tabref}[1]{Tab.~\ref{tab:#1}\xspace}
\newcommand{\Tabref}[1]{Table~\ref{tab:#1}\xspace}
\newcommand{\figref}[1]{Fig.~\ref{fig:#1}\xspace}
\newcommand{\Figref}[1]{Figure~\ref{fig:#1}\xspace}
\renewcommand{\eqref}[1]{(\ref{#1})\xspace}
\newcommand{\Eqref}[1]{Equation~\ref{eq:#1}\xspace}

%%%%
%%%%    Mathematical Symbols
%%%%

\newcommand{\goesto}{\ensuremath{\rightarrow}}
\newcommand{\one}{\ensuremath{\mathbbm{1}}}

%%%%
%%%%    Physical Quantities and Constants
%%%%


%%%%
%%%%    Software
%%%%

\newcommand{\git}{\texttt{git}\xspace}

% Put an xspace after \LaTeX:
\let\builtinLaTeX\LaTeX
\def\LaTeX{\builtinLaTeX\xspace}
 % input rather than include so we don't create macros.aux

%%%%
%%%%    Document preparation
%%%%


\begin{document}

\title{Example \LaTeX + \texttt{git} }

\author{Evan Berkowitz}

\date{\today}

\begin{abstract}
Here's a bare-bones example where I set up a git repo with all the \LaTeX-related things that I like.
\vfill % And below is where you see the state of the repo that produced this pdf.
\input{git_information}
\end{abstract}

\maketitle

\section{Introduction}

\LaTeX is a system for producing beautiful documents.
It takes as input a set of plain text files, and produces a variety of output types; the most useful to me is PDF.

\git is a decentralized version control system.
It's fantastic.

This repo is an example that is supposed to make it easy to make \git-controlled \LaTeX documents.
In particular, I have written \git hooks that ensure individual commits successfully produce a PDF and ensure that after a pull the PDF is automatically recompiled with the latest edits.
That helps keep everybody\ldots\ on the same page.

I also have a small shell script that reports the status of the repo when the PDF is being compiled that can be immediately incorporated into the \LaTeX document.
The script \texttt{git\_information.sh} produces \texttt{git\_information.tex} which looks, in the current example, like
\begin{verbatim}
% Automatically generated by git_information.sh
1 files different from commit d50192b from 2018-08-24 10:26:28 +0200
\end{verbatim}
The one dirty file is the one I'm working on---\texttt{section/introduction.tex}.
To get the \git status I simply do \texttt{\textbackslash{}input\{git\_information\}}.
This is included in the left margin of the \texttt{draft.pdf} target.  The \texttt{master.pdf} target simply sets \texttt{git\_information.tex} to the empty file.

\section{\texttt{git} Hooks}

\git hooks are scripts that you inject into the \git workflow.
They can be installed into the repo with
\begin{verbatim}
    make git-hooks
\end{verbatim}
which symbolically links the hooks in the \texttt{hooks} directory into the \texttt{.git/hooks} directory.

The \texttt{pre-commit} hook prevents the user from \texttt{git commit}ting a commit that doesn't compile, which it tests for by stashing all files that don't belong to the commit and trying to \texttt{make master.pdf}.  The output of that \texttt{make} is created in the \texttt{.git-pre-commit-hook.log} in the root directory of this repository, so that the user can see the \LaTeX in the event the PDF doesn't compile.

The \texttt{post-merge} hook tries to compile the latest PDF, also using \texttt{make master.pdf} after the repository has been updated.
It is much less sophisticated than the pre-commit hook, because (under the assumption the pre-commit hook is working) the committed state of the repo should always compile.

\section{Makefile}

The provided makefile uses \texttt{pdflatex} and \texttt{bibtex} to compile \texttt{master.pdf}, the base of which is set in the MASTER variable.

The \texttt{\$(MASTER).pdf} target depends on the files in the \texttt{section} directory, \texttt{macros.tex}, and \texttt{\$(MASTER).tex}.

The \texttt{\$(GIT\_STATUS)} phony target uses \texttt{git\_information.sh} to produce the \texttt{git\_information.tex} file explained above.

The \texttt{git-hooks} target symbolically links the git hooks into the repository's \texttt{.git} directory.

The \texttt{clean\_temporary\_files} and \texttt{clean} targets remove a lot of cruft automatically generated during the course of compiling the \LaTeX to PDF.

The \texttt{watch} target uses \texttt{watchman-make}\cite{watchman} to continuously update the PDF as you make changes to the source files.
This can interfere with the pre-commit git hook.
It can also get stuck in an infinite loop.
But if you tend to write decent \LaTeX the first time around, it'll be pretty reliable.



\bibliography{master}

\end{document}