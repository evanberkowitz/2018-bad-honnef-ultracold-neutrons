\section{\texttt{git} Hooks}

\git hooks are scripts that you inject into the \git workflow.
They can be installed into the repo with
\begin{verbatim}
    make git-hooks
\end{verbatim}
which symbolically links the hooks in the \texttt{hooks} directory into the \texttt{.git/hooks} directory.

The \texttt{pre-commit} hook prevents the user from \texttt{git commit}ting a commit that doesn't compile, which it tests for by stashing all files that don't belong to the commit and trying to \texttt{make master.pdf}.  The output of that \texttt{make} is created in the \texttt{.git-pre-commit-hook.log} in the root directory of this repository, so that the user can see the \LaTeX in the event the PDF doesn't compile.

The \texttt{post-merge} hook tries to compile the latest PDF, also using \texttt{make master.pdf} after the repository has been updated.
It is much less sophisticated than the pre-commit hook, because (under the assumption the pre-commit hook is working) the committed state of the repo should always compile.
